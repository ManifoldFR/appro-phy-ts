
% Default to the notebook output style

    


% Inherit from the specified cell style.




    
\documentclass[11pt]{article}

    
    
    % Nicer default font (+ math font) than Computer Modern for most use cases
    %\usepackage{mathpazo}
	\usepackage{fontspec}
	\usepackage{polyglossia}
	\setdefaultlanguage{french}
    % Basic figure setup, for now with no caption control since it's done
    % automatically by Pandoc (which extracts ![](path) syntax from Markdown).
    \usepackage{graphicx}
    % We will generate all images so they have a width \maxwidth. This means
    % that they will get their normal width if they fit onto the page, but
    % are scaled down if they would overflow the margins.
    \makeatletter
    \def\maxwidth{\ifdim\Gin@nat@width>\linewidth\linewidth
    \else\Gin@nat@width\fi}
    \makeatother
    \let\Oldincludegraphics\includegraphics
    % Set max figure width to be 80% of text width, for now hardcoded.
    \renewcommand{\includegraphics}[1]{\Oldincludegraphics[width=.8\maxwidth]{#1}}
    % Ensure that by default, figures have no caption (until we provide a
    % proper Figure object with a Caption API and a way to capture that
    % in the conversion process - todo).
    \usepackage{caption}
    \DeclareCaptionLabelFormat{nolabel}{}
    \captionsetup{labelformat=nolabel}

    \usepackage{adjustbox} % Used to constrain images to a maximum size 
    \usepackage{xcolor} % Allow colors to be defined
    \usepackage{enumerate} % Needed for markdown enumerations to work
    \usepackage{geometry} % Used to adjust the document margins
    \usepackage{mathtools} % Equations
    \usepackage{amssymb} % Equations
	\usepackage{unicode-math}
	\let\mathbb\relax
	\DeclareMathAlphabet{\mathbb}{U}{msb}{m}{n}
    \usepackage{textcomp} % defines textquotesingle
    % Hack from http://tex.stackexchange.com/a/47451/13684:
    \AtBeginDocument{%
        \def\PYZsq{\textquotesingle}% Upright quotes in Pygmentized code
    }
    \usepackage{upquote} % Upright quotes for verbatim code
    \usepackage{eurosym} % defines \euro
    \usepackage{fancyvrb} % verbatim replacement that allows latex
    \usepackage{grffile} % extends the file name processing of package graphics 
                         % to support a larger range 
    % The hyperref package gives us a pdf with properly built
    % internal navigation ('pdf bookmarks' for the table of contents,
    % internal cross-reference links, web links for URLs, etc.)
    \usepackage{hyperref}
    \usepackage{longtable} % longtable support required by pandoc >1.10
    \usepackage{booktabs}  % table support for pandoc > 1.12.2
    \usepackage[inline]{enumitem} % IRkernel/repr support (it uses the enumerate* environment)
    \usepackage[normalem]{ulem} % ulem is needed to support strikethroughs (\sout)
                                % normalem makes italics be italics, not underlines
    

    
    
    % Colors for the hyperref package
    \definecolor{urlcolor}{rgb}{0,.145,.698}
    \definecolor{linkcolor}{rgb}{.71,0.21,0.01}
    \definecolor{citecolor}{rgb}{.12,.54,.11}

    % ANSI colors
    \definecolor{ansi-black}{HTML}{3E424D}
    \definecolor{ansi-black-intense}{HTML}{282C36}
    \definecolor{ansi-red}{HTML}{E75C58}
    \definecolor{ansi-red-intense}{HTML}{B22B31}
    \definecolor{ansi-green}{HTML}{00A250}
    \definecolor{ansi-green-intense}{HTML}{007427}
    \definecolor{ansi-yellow}{HTML}{DDB62B}
    \definecolor{ansi-yellow-intense}{HTML}{B27D12}
    \definecolor{ansi-blue}{HTML}{208FFB}
    \definecolor{ansi-blue-intense}{HTML}{0065CA}
    \definecolor{ansi-magenta}{HTML}{D160C4}
    \definecolor{ansi-magenta-intense}{HTML}{A03196}
    \definecolor{ansi-cyan}{HTML}{60C6C8}
    \definecolor{ansi-cyan-intense}{HTML}{258F8F}
    \definecolor{ansi-white}{HTML}{C5C1B4}
    \definecolor{ansi-white-intense}{HTML}{A1A6B2}

    % commands and environments needed by pandoc snippets
    % extracted from the output of `pandoc -s`
    \providecommand{\tightlist}{%
      \setlength{\itemsep}{0pt}\setlength{\parskip}{0pt}}
    \DefineVerbatimEnvironment{Highlighting}{Verbatim}{commandchars=\\\{\}}
    % Add ',fontsize=\small' for more characters per line
    \newenvironment{Shaded}{}{}
    \newcommand{\KeywordTok}[1]{\textcolor[rgb]{0.00,0.44,0.13}{\textbf{{#1}}}}
    \newcommand{\DataTypeTok}[1]{\textcolor[rgb]{0.56,0.13,0.00}{{#1}}}
    \newcommand{\DecValTok}[1]{\textcolor[rgb]{0.25,0.63,0.44}{{#1}}}
    \newcommand{\BaseNTok}[1]{\textcolor[rgb]{0.25,0.63,0.44}{{#1}}}
    \newcommand{\FloatTok}[1]{\textcolor[rgb]{0.25,0.63,0.44}{{#1}}}
    \newcommand{\CharTok}[1]{\textcolor[rgb]{0.25,0.44,0.63}{{#1}}}
    \newcommand{\StringTok}[1]{\textcolor[rgb]{0.25,0.44,0.63}{{#1}}}
    \newcommand{\CommentTok}[1]{\textcolor[rgb]{0.38,0.63,0.69}{\textit{{#1}}}}
    \newcommand{\OtherTok}[1]{\textcolor[rgb]{0.00,0.44,0.13}{{#1}}}
    \newcommand{\AlertTok}[1]{\textcolor[rgb]{1.00,0.00,0.00}{\textbf{{#1}}}}
    \newcommand{\FunctionTok}[1]{\textcolor[rgb]{0.02,0.16,0.49}{{#1}}}
    \newcommand{\RegionMarkerTok}[1]{{#1}}
    \newcommand{\ErrorTok}[1]{\textcolor[rgb]{1.00,0.00,0.00}{\textbf{{#1}}}}
    \newcommand{\NormalTok}[1]{{#1}}
    
    % Additional commands for more recent versions of Pandoc
    \newcommand{\ConstantTok}[1]{\textcolor[rgb]{0.53,0.00,0.00}{{#1}}}
    \newcommand{\SpecialCharTok}[1]{\textcolor[rgb]{0.25,0.44,0.63}{{#1}}}
    \newcommand{\VerbatimStringTok}[1]{\textcolor[rgb]{0.25,0.44,0.63}{{#1}}}
    \newcommand{\SpecialStringTok}[1]{\textcolor[rgb]{0.73,0.40,0.53}{{#1}}}
    \newcommand{\ImportTok}[1]{{#1}}
    \newcommand{\DocumentationTok}[1]{\textcolor[rgb]{0.73,0.13,0.13}{\textit{{#1}}}}
    \newcommand{\AnnotationTok}[1]{\textcolor[rgb]{0.38,0.63,0.69}{\textbf{\textit{{#1}}}}}
    \newcommand{\CommentVarTok}[1]{\textcolor[rgb]{0.38,0.63,0.69}{\textbf{\textit{{#1}}}}}
    \newcommand{\VariableTok}[1]{\textcolor[rgb]{0.10,0.09,0.49}{{#1}}}
    \newcommand{\ControlFlowTok}[1]{\textcolor[rgb]{0.00,0.44,0.13}{\textbf{{#1}}}}
    \newcommand{\OperatorTok}[1]{\textcolor[rgb]{0.40,0.40,0.40}{{#1}}}
    \newcommand{\BuiltInTok}[1]{{#1}}
    \newcommand{\ExtensionTok}[1]{{#1}}
    \newcommand{\PreprocessorTok}[1]{\textcolor[rgb]{0.74,0.48,0.00}{{#1}}}
    \newcommand{\AttributeTok}[1]{\textcolor[rgb]{0.49,0.56,0.16}{{#1}}}
    \newcommand{\InformationTok}[1]{\textcolor[rgb]{0.38,0.63,0.69}{\textbf{\textit{{#1}}}}}
    \newcommand{\WarningTok}[1]{\textcolor[rgb]{0.38,0.63,0.69}{\textbf{\textit{{#1}}}}}
    
    
    % Define a nice break command that doesn't care if a line doesn't already
    % exist.
    \def\br{\hspace*{\fill} \\* }
    % Math Jax compatability definitions
    \def\gt{>}
    \def\lt{<}
    % Document parameters
    \title{variable\_current\_field}
    
    
    

    % Pygments definitions
    
\makeatletter
\def\PY@reset{\let\PY@it=\relax \let\PY@bf=\relax%
    \let\PY@ul=\relax \let\PY@tc=\relax%
    \let\PY@bc=\relax \let\PY@ff=\relax}
\def\PY@tok#1{\csname PY@tok@#1\endcsname}
\def\PY@toks#1+{\ifx\relax#1\empty\else%
    \PY@tok{#1}\expandafter\PY@toks\fi}
\def\PY@do#1{\PY@bc{\PY@tc{\PY@ul{%
    \PY@it{\PY@bf{\PY@ff{#1}}}}}}}
\def\PY#1#2{\PY@reset\PY@toks#1+\relax+\PY@do{#2}}

\expandafter\def\csname PY@tok@cm\endcsname{\let\PY@it=\textit\def\PY@tc##1{\textcolor[rgb]{0.25,0.50,0.50}{##1}}}
\expandafter\def\csname PY@tok@mb\endcsname{\def\PY@tc##1{\textcolor[rgb]{0.40,0.40,0.40}{##1}}}
\expandafter\def\csname PY@tok@gu\endcsname{\let\PY@bf=\textbf\def\PY@tc##1{\textcolor[rgb]{0.50,0.00,0.50}{##1}}}
\expandafter\def\csname PY@tok@sc\endcsname{\def\PY@tc##1{\textcolor[rgb]{0.73,0.13,0.13}{##1}}}
\expandafter\def\csname PY@tok@w\endcsname{\def\PY@tc##1{\textcolor[rgb]{0.73,0.73,0.73}{##1}}}
\expandafter\def\csname PY@tok@c\endcsname{\let\PY@it=\textit\def\PY@tc##1{\textcolor[rgb]{0.25,0.50,0.50}{##1}}}
\expandafter\def\csname PY@tok@ge\endcsname{\let\PY@it=\textit}
\expandafter\def\csname PY@tok@ss\endcsname{\def\PY@tc##1{\textcolor[rgb]{0.10,0.09,0.49}{##1}}}
\expandafter\def\csname PY@tok@se\endcsname{\let\PY@bf=\textbf\def\PY@tc##1{\textcolor[rgb]{0.73,0.40,0.13}{##1}}}
\expandafter\def\csname PY@tok@gt\endcsname{\def\PY@tc##1{\textcolor[rgb]{0.00,0.27,0.87}{##1}}}
\expandafter\def\csname PY@tok@vc\endcsname{\def\PY@tc##1{\textcolor[rgb]{0.10,0.09,0.49}{##1}}}
\expandafter\def\csname PY@tok@vi\endcsname{\def\PY@tc##1{\textcolor[rgb]{0.10,0.09,0.49}{##1}}}
\expandafter\def\csname PY@tok@gd\endcsname{\def\PY@tc##1{\textcolor[rgb]{0.63,0.00,0.00}{##1}}}
\expandafter\def\csname PY@tok@kp\endcsname{\def\PY@tc##1{\textcolor[rgb]{0.00,0.50,0.00}{##1}}}
\expandafter\def\csname PY@tok@nc\endcsname{\let\PY@bf=\textbf\def\PY@tc##1{\textcolor[rgb]{0.00,0.00,1.00}{##1}}}
\expandafter\def\csname PY@tok@sr\endcsname{\def\PY@tc##1{\textcolor[rgb]{0.73,0.40,0.53}{##1}}}
\expandafter\def\csname PY@tok@nl\endcsname{\def\PY@tc##1{\textcolor[rgb]{0.63,0.63,0.00}{##1}}}
\expandafter\def\csname PY@tok@sh\endcsname{\def\PY@tc##1{\textcolor[rgb]{0.73,0.13,0.13}{##1}}}
\expandafter\def\csname PY@tok@kt\endcsname{\def\PY@tc##1{\textcolor[rgb]{0.69,0.00,0.25}{##1}}}
\expandafter\def\csname PY@tok@gh\endcsname{\let\PY@bf=\textbf\def\PY@tc##1{\textcolor[rgb]{0.00,0.00,0.50}{##1}}}
\expandafter\def\csname PY@tok@nv\endcsname{\def\PY@tc##1{\textcolor[rgb]{0.10,0.09,0.49}{##1}}}
\expandafter\def\csname PY@tok@kd\endcsname{\let\PY@bf=\textbf\def\PY@tc##1{\textcolor[rgb]{0.00,0.50,0.00}{##1}}}
\expandafter\def\csname PY@tok@ni\endcsname{\let\PY@bf=\textbf\def\PY@tc##1{\textcolor[rgb]{0.60,0.60,0.60}{##1}}}
\expandafter\def\csname PY@tok@s1\endcsname{\def\PY@tc##1{\textcolor[rgb]{0.73,0.13,0.13}{##1}}}
\expandafter\def\csname PY@tok@err\endcsname{\def\PY@bc##1{\setlength{\fboxsep}{0pt}\fcolorbox[rgb]{1.00,0.00,0.00}{1,1,1}{\strut ##1}}}
\expandafter\def\csname PY@tok@no\endcsname{\def\PY@tc##1{\textcolor[rgb]{0.53,0.00,0.00}{##1}}}
\expandafter\def\csname PY@tok@vg\endcsname{\def\PY@tc##1{\textcolor[rgb]{0.10,0.09,0.49}{##1}}}
\expandafter\def\csname PY@tok@nn\endcsname{\let\PY@bf=\textbf\def\PY@tc##1{\textcolor[rgb]{0.00,0.00,1.00}{##1}}}
\expandafter\def\csname PY@tok@nd\endcsname{\def\PY@tc##1{\textcolor[rgb]{0.67,0.13,1.00}{##1}}}
\expandafter\def\csname PY@tok@cs\endcsname{\let\PY@it=\textit\def\PY@tc##1{\textcolor[rgb]{0.25,0.50,0.50}{##1}}}
\expandafter\def\csname PY@tok@nf\endcsname{\def\PY@tc##1{\textcolor[rgb]{0.00,0.00,1.00}{##1}}}
\expandafter\def\csname PY@tok@k\endcsname{\let\PY@bf=\textbf\def\PY@tc##1{\textcolor[rgb]{0.00,0.50,0.00}{##1}}}
\expandafter\def\csname PY@tok@nb\endcsname{\def\PY@tc##1{\textcolor[rgb]{0.00,0.50,0.00}{##1}}}
\expandafter\def\csname PY@tok@mf\endcsname{\def\PY@tc##1{\textcolor[rgb]{0.40,0.40,0.40}{##1}}}
\expandafter\def\csname PY@tok@ne\endcsname{\let\PY@bf=\textbf\def\PY@tc##1{\textcolor[rgb]{0.82,0.25,0.23}{##1}}}
\expandafter\def\csname PY@tok@kc\endcsname{\let\PY@bf=\textbf\def\PY@tc##1{\textcolor[rgb]{0.00,0.50,0.00}{##1}}}
\expandafter\def\csname PY@tok@sx\endcsname{\def\PY@tc##1{\textcolor[rgb]{0.00,0.50,0.00}{##1}}}
\expandafter\def\csname PY@tok@ow\endcsname{\let\PY@bf=\textbf\def\PY@tc##1{\textcolor[rgb]{0.67,0.13,1.00}{##1}}}
\expandafter\def\csname PY@tok@si\endcsname{\let\PY@bf=\textbf\def\PY@tc##1{\textcolor[rgb]{0.73,0.40,0.53}{##1}}}
\expandafter\def\csname PY@tok@sd\endcsname{\let\PY@it=\textit\def\PY@tc##1{\textcolor[rgb]{0.73,0.13,0.13}{##1}}}
\expandafter\def\csname PY@tok@sb\endcsname{\def\PY@tc##1{\textcolor[rgb]{0.73,0.13,0.13}{##1}}}
\expandafter\def\csname PY@tok@gs\endcsname{\let\PY@bf=\textbf}
\expandafter\def\csname PY@tok@na\endcsname{\def\PY@tc##1{\textcolor[rgb]{0.49,0.56,0.16}{##1}}}
\expandafter\def\csname PY@tok@bp\endcsname{\def\PY@tc##1{\textcolor[rgb]{0.00,0.50,0.00}{##1}}}
\expandafter\def\csname PY@tok@gp\endcsname{\let\PY@bf=\textbf\def\PY@tc##1{\textcolor[rgb]{0.00,0.00,0.50}{##1}}}
\expandafter\def\csname PY@tok@c1\endcsname{\let\PY@it=\textit\def\PY@tc##1{\textcolor[rgb]{0.25,0.50,0.50}{##1}}}
\expandafter\def\csname PY@tok@mi\endcsname{\def\PY@tc##1{\textcolor[rgb]{0.40,0.40,0.40}{##1}}}
\expandafter\def\csname PY@tok@m\endcsname{\def\PY@tc##1{\textcolor[rgb]{0.40,0.40,0.40}{##1}}}
\expandafter\def\csname PY@tok@gi\endcsname{\def\PY@tc##1{\textcolor[rgb]{0.00,0.63,0.00}{##1}}}
\expandafter\def\csname PY@tok@kr\endcsname{\let\PY@bf=\textbf\def\PY@tc##1{\textcolor[rgb]{0.00,0.50,0.00}{##1}}}
\expandafter\def\csname PY@tok@nt\endcsname{\let\PY@bf=\textbf\def\PY@tc##1{\textcolor[rgb]{0.00,0.50,0.00}{##1}}}
\expandafter\def\csname PY@tok@cp\endcsname{\def\PY@tc##1{\textcolor[rgb]{0.74,0.48,0.00}{##1}}}
\expandafter\def\csname PY@tok@il\endcsname{\def\PY@tc##1{\textcolor[rgb]{0.40,0.40,0.40}{##1}}}
\expandafter\def\csname PY@tok@mh\endcsname{\def\PY@tc##1{\textcolor[rgb]{0.40,0.40,0.40}{##1}}}
\expandafter\def\csname PY@tok@gr\endcsname{\def\PY@tc##1{\textcolor[rgb]{1.00,0.00,0.00}{##1}}}
\expandafter\def\csname PY@tok@kn\endcsname{\let\PY@bf=\textbf\def\PY@tc##1{\textcolor[rgb]{0.00,0.50,0.00}{##1}}}
\expandafter\def\csname PY@tok@o\endcsname{\def\PY@tc##1{\textcolor[rgb]{0.40,0.40,0.40}{##1}}}
\expandafter\def\csname PY@tok@s\endcsname{\def\PY@tc##1{\textcolor[rgb]{0.73,0.13,0.13}{##1}}}
\expandafter\def\csname PY@tok@s2\endcsname{\def\PY@tc##1{\textcolor[rgb]{0.73,0.13,0.13}{##1}}}
\expandafter\def\csname PY@tok@mo\endcsname{\def\PY@tc##1{\textcolor[rgb]{0.40,0.40,0.40}{##1}}}
\expandafter\def\csname PY@tok@go\endcsname{\def\PY@tc##1{\textcolor[rgb]{0.53,0.53,0.53}{##1}}}
\expandafter\def\csname PY@tok@cpf\endcsname{\let\PY@it=\textit\def\PY@tc##1{\textcolor[rgb]{0.25,0.50,0.50}{##1}}}
\expandafter\def\csname PY@tok@ch\endcsname{\let\PY@it=\textit\def\PY@tc##1{\textcolor[rgb]{0.25,0.50,0.50}{##1}}}

\def\PYZbs{\char`\\}
\def\PYZus{\char`\_}
\def\PYZob{\char`\{}
\def\PYZcb{\char`\}}
\def\PYZca{\char`\^}
\def\PYZam{\char`\&}
\def\PYZlt{\char`\<}
\def\PYZgt{\char`\>}
\def\PYZsh{\char`\#}
\def\PYZpc{\char`\%}
\def\PYZdl{\char`\$}
\def\PYZhy{\char`\-}
\def\PYZsq{\char`\'}
\def\PYZdq{\char`\"}
\def\PYZti{\char`\~}
% for compatibility with earlier versions
\def\PYZat{@}
\def\PYZlb{[}
\def\PYZrb{]}
\makeatother


    % Exact colors from NB
    \definecolor{incolor}{rgb}{0.0, 0.0, 0.5}
    \definecolor{outcolor}{rgb}{0.545, 0.0, 0.0}



    
    % Prevent overflowing lines due to hard-to-break entities
    \sloppy 
    % Setup hyperref package
    \hypersetup{
      breaklinks=true,  % so long urls are correctly broken across lines
      colorlinks=true,
      urlcolor=urlcolor,
      linkcolor=linkcolor,
      citecolor=citecolor,
      }
    % Slightly bigger margins than the latex defaults
    
    \geometry{verbose,tmargin=1in,bmargin=1in,lmargin=1in,rmargin=1in}
    
    

    \begin{document}
    
    
    %\maketitle
    
    

    
    \begin{Verbatim}[commandchars=\\\{\}]
{\color{incolor}In [{\color{incolor}213}]:} \PY{o}{\PYZpc{}}\PY{k}{matplotlib} inline
          \PY{k+kn}{import} \PY{n+nn}{matplotlib} \PY{k}{as} \PY{n+nn}{mpl}
          \PY{k+kn}{import} \PY{n+nn}{matplotlib}\PY{n+nn}{.}\PY{n+nn}{colors} \PY{k}{as} \PY{n+nn}{colors}
          \PY{k+kn}{import} \PY{n+nn}{matplotlib}\PY{n+nn}{.}\PY{n+nn}{pyplot} \PY{k}{as} \PY{n+nn}{plt}
          
          \PY{k+kn}{import} \PY{n+nn}{numpy} \PY{k}{as} \PY{n+nn}{np}
          \PY{k+kn}{import} \PY{n+nn}{scipy}\PY{n+nn}{.}\PY{n+nn}{special} \PY{k}{as} \PY{n+nn}{spec}
          \PY{k+kn}{import} \PY{n+nn}{scipy}\PY{n+nn}{.}\PY{n+nn}{integrate} \PY{k}{as} \PY{n+nn}{inte}
          
          \PY{k+kn}{import} \PY{n+nn}{sympy} \PY{k}{as} \PY{n+nn}{sp}
          
          \PY{n}{sp}\PY{o}{.}\PY{n}{init\PYZus{}printing}\PY{p}{(}\PY{p}{)}
\end{Verbatim}

    \section{Position du problème}\label{position-du-probluxe8me}

    \section{\texorpdfstring{Construction du champ magnétique
\(\mathbf{B}\)}{Construction du champ magnétique \textbackslash{}mathbf\{B\}}}\label{construction-du-champ-magnuxe9tique-mathbfb}

    \begin{Verbatim}[commandchars=\\\{\}]
{\color{incolor}In [{\color{incolor}258}]:} \PY{n}{r} \PY{o}{=} \PY{n}{sp}\PY{o}{.}\PY{n}{symbols}\PY{p}{(}\PY{l+s+s1}{\PYZsq{}}\PY{l+s+s1}{r}\PY{l+s+s1}{\PYZsq{}}\PY{p}{)}
          \PY{n}{t} \PY{o}{=} \PY{n}{sp}\PY{o}{.}\PY{n}{symbols}\PY{p}{(}\PY{l+s+s1}{\PYZsq{}}\PY{l+s+s1}{t}\PY{l+s+s1}{\PYZsq{}}\PY{p}{)}
          \PY{n}{omega}\PY{p}{,} \PY{n}{phi} \PY{o}{=} \PY{n}{sp}\PY{o}{.}\PY{n}{symbols}\PY{p}{(}\PY{l+s+s1}{\PYZsq{}}\PY{l+s+s1}{omega phi}\PY{l+s+s1}{\PYZsq{}}\PY{p}{)}
          
          \PY{n}{c} \PY{o}{=} \PY{l+m+mi}{3}\PY{n}{e8}      \PY{c+c1}{\PYZsh{} Célérité de la lumière dans le vide}
          \PY{n}{k} \PY{o}{=} \PY{n}{omega}\PY{o}{/}\PY{n}{c}  \PY{c+c1}{\PYZsh{} Nombre d\PYZsq{}onde}
          
          \PY{c+c1}{\PYZsh{} Composante du champ magnétique associée à la pulsation omega}
          \PY{n}{B\PYZus{}component} \PY{o}{=} \PY{p}{(}\PY{n}{sp}\PY{o}{.}\PY{n}{besselj}\PY{p}{(}\PY{l+m+mi}{0}\PY{p}{,}\PY{n}{k}\PY{o}{*}\PY{n}{r}\PY{p}{)}\PY{o}{\PYZhy{}}\PY{n}{sp}\PY{o}{.}\PY{n}{bessely}\PY{p}{(}\PY{l+m+mi}{0}\PY{p}{,}\PY{n}{k}\PY{o}{*}\PY{n}{r}\PY{p}{)}\PY{o}{/}\PY{p}{(}\PY{l+m+mi}{1}\PY{o}{+}\PY{n}{sp}\PY{o}{.}\PY{n}{exp}\PY{p}{(}\PY{o}{\PYZhy{}}\PY{n}{omega}\PY{o}{*}\PY{l+m+mi}{1}\PY{n}{e}\PY{o}{\PYZhy{}}\PY{l+m+mi}{5}\PY{p}{)}\PY{p}{)}\PY{p}{)} \PYZbs{}
              \PY{o}{*}\PY{n}{sp}\PY{o}{.}\PY{n}{cos}\PY{p}{(}\PY{n}{omega}\PY{o}{*}\PY{n}{t}\PY{o}{\PYZhy{}}\PY{n}{phi}\PY{p}{)}
          
          \PY{c+c1}{\PYZsh{} Création du champ (expression et fonction) en sommant les composantes}
          \PY{k}{def} \PY{n+nf}{create\PYZus{}Bfield}\PY{p}{(}\PY{n}{puls}\PY{p}{,}\PY{n}{phas}\PY{p}{)}\PY{p}{:}
              \PY{n}{spectr} \PY{o}{=} \PY{n+nb}{zip}\PY{p}{(}\PY{n}{puls}\PY{p}{,}\PY{n}{phas}\PY{p}{)}
              \PY{n}{B\PYZus{}field} \PY{o}{=} \PY{n+nb}{sum}\PY{p}{(}\PY{p}{[}\PY{n}{B\PYZus{}component}\PY{o}{.}\PY{n}{subs}\PY{p}{(}\PY{p}{\PYZob{}}\PY{n}{omega}\PY{p}{:}\PY{n}{om}\PY{p}{,} \PY{n}{phi}\PY{p}{:}\PY{n}{ph}\PY{p}{\PYZcb{}}\PY{p}{)} \PY{k}{for} \PY{p}{(}\PY{n}{om}\PY{p}{,}\PY{n}{ph}\PY{p}{)} \PY{o+ow}{in} \PY{n}{spectr}\PY{p}{]}\PY{p}{)}
              \PY{n}{B\PYZus{}function} \PY{o}{=} \PY{n}{sp}\PY{o}{.}\PY{n}{lambdify}\PY{p}{(}\PY{p}{(}\PY{n}{r}\PY{p}{,} \PY{n}{t}\PY{p}{)}\PY{p}{,} \PY{n}{B\PYZus{}field}\PY{p}{,} 
                  \PY{n}{modules}\PY{o}{=}\PY{p}{[}\PY{l+s+s1}{\PYZsq{}}\PY{l+s+s1}{numpy}\PY{l+s+s1}{\PYZsq{}}\PY{p}{,}\PY{p}{\PYZob{}}\PY{l+s+s2}{\PYZdq{}}\PY{l+s+s2}{besselj}\PY{l+s+s2}{\PYZdq{}}\PY{p}{:}\PY{n}{spec}\PY{o}{.}\PY{n}{jn}\PY{p}{,} \PY{l+s+s2}{\PYZdq{}}\PY{l+s+s2}{bessely}\PY{l+s+s2}{\PYZdq{}}\PY{p}{:}\PY{n}{spec}\PY{o}{.}\PY{n}{yn}\PY{p}{\PYZcb{}}\PY{p}{]}\PY{p}{)}
              \PY{k}{return} \PY{n}{B\PYZus{}field}\PY{p}{,} \PY{n}{B\PYZus{}function}
\end{Verbatim}

    \begin{Verbatim}[commandchars=\\\{\}]
{\color{incolor}In [{\color{incolor}312}]:} \PY{k}{def} \PY{n+nf}{graphe\PYZus{}B}\PY{p}{(}\PY{n}{times}\PY{p}{)}\PY{p}{:}
              \PY{l+s+sd}{\PYZsq{}\PYZsq{}\PYZsq{}}
          \PY{l+s+sd}{    Construit les graphes du champ magnétique B aux temps donnés dans la liste}
          \PY{l+s+sd}{    \PYZdq{}times\PYZdq{}}
          \PY{l+s+sd}{    \PYZsq{}\PYZsq{}\PYZsq{}}
              \PY{n}{radii} \PY{o}{=} \PY{n}{np}\PY{o}{.}\PY{n}{linspace}\PY{p}{(}\PY{n}{rmin}\PY{p}{,} \PY{n}{rmax}\PY{p}{,} \PY{l+m+mi}{10}\PY{o}{*}\PY{n}{rmax}\PY{p}{)}
              
              \PY{n}{fig}\PY{p}{,} \PY{n}{ax} \PY{o}{=} \PY{n}{plt}\PY{o}{.}\PY{n}{subplots}\PY{p}{(}\PY{l+m+mi}{1}\PY{p}{,}\PY{l+m+mi}{1}\PY{p}{,} \PY{n}{figsize}\PY{o}{=}\PY{p}{(}\PY{l+m+mi}{8}\PY{p}{,}\PY{l+m+mi}{5}\PY{p}{)}\PY{p}{,} \PY{n}{dpi}\PY{o}{=}\PY{l+m+mi}{400}\PY{p}{)}
              
              \PY{k}{if} \PY{n+nb}{hasattr}\PY{p}{(}\PY{n}{times}\PY{p}{,} \PY{l+s+s1}{\PYZsq{}}\PY{l+s+s1}{\PYZus{}\PYZus{}iter\PYZus{}\PYZus{}}\PY{l+s+s1}{\PYZsq{}}\PY{p}{)}\PY{p}{:}
                  \PY{k}{for} \PY{n}{ti} \PY{o+ow}{in} \PY{n}{times}\PY{p}{:}
                      \PY{n}{leg} \PY{o}{=} \PY{l+s+s1}{r\PYZsq{}}\PY{l+s+s1}{\PYZdl{}t= }\PY{l+s+si}{\PYZob{}:g\PYZcb{}}\PY{l+s+s1}{\PYZdl{}}\PY{l+s+s1}{\PYZsq{}}\PY{o}{.}\PY{n}{format}\PY{p}{(}\PY{n}{ti}\PY{p}{)}
                      \PY{n}{leg} \PY{o}{=} \PY{n}{leg} \PY{o}{+} \PY{l+s+s2}{r\PYZdq{}}\PY{l+s+s2}{\PYZdl{}}\PY{l+s+s2}{\PYZbs{}}\PY{l+s+s2}{ }\PY{l+s+s2}{\PYZbs{}}\PY{l+s+s2}{mathrm}\PY{l+s+si}{\PYZob{}s\PYZcb{}}\PY{l+s+s2}{\PYZdl{}}\PY{l+s+s2}{\PYZdq{}}
                      \PY{n}{champ} \PY{o}{=} \PY{n}{B\PYZus{}function}\PY{p}{(}\PY{n}{radii}\PY{p}{,} \PY{n}{ti}\PY{p}{)}
                      \PY{n}{ax}\PY{o}{.}\PY{n}{plot}\PY{p}{(}\PY{n}{radii}\PY{p}{,} \PY{n}{champ}\PY{p}{,} \PY{n}{label}\PY{o}{=}\PY{n}{leg}\PY{p}{)}
              \PY{k}{else}\PY{p}{:}
                  \PY{n}{champ} \PY{o}{=} \PY{n}{B\PYZus{}function}\PY{p}{(}\PY{n}{radii}\PY{p}{,} \PY{n}{ti}\PY{p}{)}
                  \PY{n}{leg} \PY{o}{=} \PY{l+s+s1}{r\PYZsq{}}\PY{l+s+s1}{\PYZdl{}t = }\PY{l+s+si}{\PYZob{}:g\PYZcb{}}\PY{l+s+s1}{\PYZdl{}}\PY{l+s+s1}{\PYZsq{}}\PY{o}{.}\PY{n}{format}\PY{p}{(}\PY{n}{ti}\PY{p}{)}
                  \PY{n}{leg} \PY{o}{=} \PY{n}{leg} \PY{o}{+} \PY{l+s+s2}{r\PYZdq{}}\PY{l+s+s2}{\PYZdl{}}\PY{l+s+s2}{\PYZbs{}}\PY{l+s+s2}{ }\PY{l+s+s2}{\PYZbs{}}\PY{l+s+s2}{mathrm}\PY{l+s+si}{\PYZob{}s\PYZcb{}}\PY{l+s+s2}{\PYZdl{}}\PY{l+s+s2}{\PYZdq{}}
                  \PY{n}{ax}\PY{o}{.}\PY{n}{plot}\PY{p}{(}\PY{n}{radii}\PY{p}{,} \PY{n}{champ}\PY{p}{,} \PY{n}{label}\PY{o}{=}\PY{n}{leg}\PY{p}{)}
              
              \PY{n}{ax}\PY{o}{.}\PY{n}{grid}\PY{p}{(}\PY{p}{)}
              \PY{n}{ax}\PY{o}{.}\PY{n}{legend}\PY{p}{(}\PY{p}{)}
              \PY{n}{ax}\PY{o}{.}\PY{n}{set\PYZus{}xlabel}\PY{p}{(}\PY{l+s+s2}{\PYZdq{}}\PY{l+s+s2}{Distance \PYZdl{}r\PYZdl{} (m)}\PY{l+s+s2}{\PYZdq{}}\PY{p}{)}
              \PY{n}{ax}\PY{o}{.}\PY{n}{set\PYZus{}ylabel}\PY{p}{(}\PY{l+s+s2}{\PYZdq{}}\PY{l+s+s2}{Valeur du champ (T)}\PY{l+s+s2}{\PYZdq{}}\PY{p}{)}
              \PY{n}{ax}\PY{o}{.}\PY{n}{set\PYZus{}title}\PY{p}{(}\PY{l+s+s1}{r\PYZsq{}}\PY{l+s+s1}{Champ magnétique }\PY{l+s+s1}{\PYZsq{}} \PY{o}{+} \PY{l+s+s1}{r\PYZsq{}}\PY{l+s+s1}{\PYZdl{}}\PY{l+s+s1}{\PYZbs{}}\PY{l+s+s1}{mathbf}\PY{l+s+si}{\PYZob{}B\PYZcb{}}\PY{l+s+s1}{\PYZdl{}}\PY{l+s+s1}{\PYZsq{}} \PYZbs{}
                           \PY{o}{+} \PY{l+s+s1}{\PYZsq{}}\PY{l+s+s1}{ créé par un courant variable}\PY{l+s+s1}{\PYZsq{}}\PY{p}{)}
              
              \PY{n}{fig}\PY{o}{.}\PY{n}{tight\PYZus{}layout}\PY{p}{(}\PY{p}{)}
              
              \PY{k}{return} \PY{n}{fig}\PY{p}{,} \PY{n}{ax}
\end{Verbatim}

    \begin{Verbatim}[commandchars=\\\{\}]
{\color{incolor}In [{\color{incolor}285}]:} \PY{k}{def} \PY{n+nf}{build\PYZus{}field}\PY{p}{(}\PY{n}{t}\PY{p}{)}\PY{p}{:}
              \PY{l+s+sd}{\PYZdq{}\PYZdq{}\PYZdq{}}
          \PY{l+s+sd}{    Portrait du champ magnétique à l\PYZsq{}instant t}
          \PY{l+s+sd}{    \PYZdq{}\PYZdq{}\PYZdq{}}
              \PY{n}{wind} \PY{o}{=} \PY{n}{rmax}
              
              \PY{n}{Y}\PY{p}{,} \PY{n}{X} \PY{o}{=} \PY{n}{np}\PY{o}{.}\PY{n}{ogrid}\PY{p}{[}\PY{o}{\PYZhy{}}\PY{n}{wind}\PY{p}{:}\PY{n}{wind}\PY{p}{:}\PY{n}{wind}\PY{o}{*}\PY{l+m+mi}{10}\PY{n}{j}\PY{p}{,} \PY{o}{\PYZhy{}}\PY{n}{wind}\PY{p}{:}\PY{n}{wind}\PY{p}{:}\PY{n}{wind}\PY{o}{*}\PY{l+m+mi}{10}\PY{n}{j}\PY{p}{]}
              
              \PY{k}{def} \PY{n+nf}{field\PYZus{}func}\PY{p}{(}\PY{n}{x}\PY{p}{,}\PY{n}{y}\PY{p}{)}\PY{p}{:}
                  \PY{n}{r} \PY{o}{=} \PY{n}{np}\PY{o}{.}\PY{n}{sqrt}\PY{p}{(}\PY{n}{x}\PY{o}{*}\PY{n}{x}\PY{o}{+}\PY{n}{y}\PY{o}{*}\PY{n}{y}\PY{p}{)}
                  \PY{n}{Btheta} \PY{o}{=} \PY{n}{B\PYZus{}function}\PY{p}{(}\PY{n}{r}\PY{p}{,} \PY{n}{t}\PY{p}{)}
                  \PY{n}{direct} \PY{o}{=} \PY{n}{np}\PY{o}{.}\PY{n}{array}\PY{p}{(}\PY{p}{[}\PY{o}{\PYZhy{}}\PY{n}{y}\PY{o}{/}\PY{n}{r}\PY{p}{,} \PY{n}{x}\PY{o}{/}\PY{n}{r}\PY{p}{]}\PY{p}{)}
                  \PY{k}{return} \PY{n}{Btheta}\PY{o}{*}\PY{n}{direct}
              
              \PY{n}{BX}\PY{p}{,} \PY{n}{BY} \PY{o}{=} \PY{n}{field\PYZus{}func}\PY{p}{(}\PY{n}{X}\PY{p}{,} \PY{n}{Y}\PY{p}{)}
              
              
              \PY{n}{fig}\PY{p}{,} \PY{n}{ax} \PY{o}{=} \PY{n}{plt}\PY{o}{.}\PY{n}{subplots}\PY{p}{(}\PY{l+m+mi}{1}\PY{p}{,} \PY{l+m+mi}{1}\PY{p}{,} \PY{n}{figsize}\PY{o}{=}\PY{p}{(}\PY{l+m+mi}{8}\PY{p}{,}\PY{l+m+mi}{8}\PY{p}{)}\PY{p}{)}
              \PY{n}{ax}\PY{o}{.}\PY{n}{grid}\PY{p}{(}\PY{k+kc}{False}\PY{p}{)}
              \PY{n}{ax}\PY{o}{.}\PY{n}{set\PYZus{}aspect}\PY{p}{(}\PY{l+s+s1}{\PYZsq{}}\PY{l+s+s1}{equal}\PY{l+s+s1}{\PYZsq{}}\PY{p}{)}
              
              \PY{n}{ax}\PY{o}{.}\PY{n}{set\PYZus{}xlim}\PY{p}{(}\PY{p}{(}\PY{o}{\PYZhy{}}\PY{n}{wind}\PY{p}{,}\PY{n}{wind}\PY{p}{)}\PY{p}{)}
              \PY{n}{ax}\PY{o}{.}\PY{n}{set\PYZus{}ylim}\PY{p}{(}\PY{p}{(}\PY{o}{\PYZhy{}}\PY{n}{wind}\PY{p}{,}\PY{n}{wind}\PY{p}{)}\PY{p}{)}
              
              \PY{n}{title\PYZus{}text} \PY{o}{=} \PY{l+s+s1}{r\PYZsq{}}\PY{l+s+s1}{Champ magnétique \PYZdl{}}\PY{l+s+s1}{\PYZbs{}}\PY{l+s+s1}{mathbf}\PY{l+s+si}{\PYZob{}B\PYZcb{}}\PY{l+s+s1}{\PYZdl{} à }\PY{l+s+s1}{\PYZsq{}}
              \PY{n}{title\PYZus{}text} \PY{o}{+}\PY{o}{=} \PY{l+s+s2}{r\PYZdq{}}\PY{l+s+s2}{\PYZdl{}t=}\PY{l+s+si}{\PYZob{}:g\PYZcb{}}\PY{l+s+s2}{\PYZdl{}}\PY{l+s+s2}{\PYZdq{}}\PY{o}{.}\PY{n}{format}\PY{p}{(}\PY{n}{t}\PY{p}{)}
              \PY{n}{title\PYZus{}text} \PY{o}{+}\PY{o}{=} \PY{l+s+s2}{r\PYZdq{}}\PY{l+s+s2}{ \PYZdl{}}\PY{l+s+s2}{\PYZbs{}}\PY{l+s+s2}{mathrm}\PY{l+s+si}{\PYZob{}s\PYZcb{}}\PY{l+s+s2}{\PYZdl{}}\PY{l+s+s2}{\PYZdq{}}
              \PY{n}{ax}\PY{o}{.}\PY{n}{set\PYZus{}title}\PY{p}{(}\PY{n}{title\PYZus{}text}\PY{p}{)}
              
              \PY{n}{intensity} \PY{o}{=} \PY{n}{np}\PY{o}{.}\PY{n}{sqrt}\PY{p}{(}\PY{n}{BX}\PY{o}{*}\PY{o}{*}\PY{l+m+mi}{2}\PY{o}{+}\PY{n}{BY}\PY{o}{*}\PY{o}{*}\PY{l+m+mi}{2}\PY{p}{)}
              \PY{n}{intensity} \PY{o}{=} \PY{n}{np}\PY{o}{.}\PY{n}{nan\PYZus{}to\PYZus{}num}\PY{p}{(}\PY{n}{intensity}\PY{p}{)}
              
              \PY{n}{heat} \PY{o}{=} \PY{n}{ax}\PY{o}{.}\PY{n}{imshow}\PY{p}{(}\PY{n}{intensity}\PY{p}{,} 
                               \PY{n}{norm}\PY{o}{=}\PY{n}{colors}\PY{o}{.}\PY{n}{LogNorm}\PY{p}{(}\PY{p}{)}\PY{p}{,} 
                               \PY{n}{extent}\PY{o}{=}\PY{p}{[}\PY{o}{\PYZhy{}}\PY{n}{wind}\PY{p}{,} \PY{n}{wind}\PY{p}{,} \PY{o}{\PYZhy{}}\PY{n}{wind}\PY{p}{,} \PY{n}{wind}\PY{p}{]}\PY{p}{,} 
                               \PY{n}{alpha}\PY{o}{=}\PY{l+m+mf}{0.6}\PY{p}{)}
              \PY{n}{cbar} \PY{o}{=} \PY{n}{fig}\PY{o}{.}\PY{n}{colorbar}\PY{p}{(}\PY{n}{heat}\PY{p}{,} \PY{n}{label}\PY{o}{=}\PY{l+s+s1}{\PYZsq{}}\PY{l+s+s1}{Intensité du champ (T)}\PY{l+s+s1}{\PYZsq{}}\PY{p}{)}
              
              \PY{n}{strm} \PY{o}{=} \PY{n}{ax}\PY{o}{.}\PY{n}{streamplot}\PY{p}{(}\PY{n}{X}\PY{p}{,}\PY{n}{Y}\PY{p}{,} \PY{n}{BX}\PY{p}{,} \PY{n}{BY}\PY{p}{,} 
                  \PY{n}{arrowstyle}\PY{o}{=}\PY{l+s+s1}{\PYZsq{}}\PY{l+s+s1}{\PYZhy{}\PYZgt{}}\PY{l+s+s1}{\PYZsq{}}\PY{p}{,} 
                  \PY{n}{color}\PY{o}{=}\PY{l+s+s1}{\PYZsq{}}\PY{l+s+s1}{k}\PY{l+s+s1}{\PYZsq{}}\PY{p}{,}
                  \PY{n}{cmap}\PY{o}{=}\PY{l+s+s1}{\PYZsq{}}\PY{l+s+s1}{inferno}\PY{l+s+s1}{\PYZsq{}}\PY{p}{,}
                  \PY{n}{linewidth}\PY{o}{=}\PY{l+m+mf}{0.8}\PY{p}{,}
                  \PY{n}{arrowsize}\PY{o}{=}\PY{l+m+mi}{2}\PY{p}{,}
                  \PY{n}{density}\PY{o}{=}\PY{l+m+mf}{1.4}\PY{p}{,}
                  \PY{p}{)}
              
              \PY{n}{fig}\PY{o}{.}\PY{n}{tight\PYZus{}layout}\PY{p}{(}\PY{p}{)}
              
              \PY{k}{return} \PY{n}{fig}
              
\end{Verbatim}

    \section{Tracés}\label{tracuxe9s}

    \subsection{Données initiales}\label{donnuxe9es-initiales}

    Entrez dans la variable \texttt{freqs} les fréquences du courant voulu,
et dans \texttt{phas} les phases associées, et exécutez la cellule
(\texttt{Ctrl\ +\ Entrée} sur le clavier) pour définir la fonction de
champ :

    \begin{Verbatim}[commandchars=\\\{\}]
{\color{incolor}In [{\color{incolor}262}]:} \PY{n}{freqs} \PY{o}{=} \PY{p}{[}\PY{n}{n}\PY{o}{*}\PY{l+m+mi}{1}\PY{n}{e7} \PY{k}{for} \PY{n}{n} \PY{o+ow}{in} \PY{n+nb}{range}\PY{p}{(}\PY{l+m+mi}{5}\PY{p}{,}\PY{l+m+mi}{7}\PY{p}{)}\PY{p}{]} \PY{o}{+} \PYZbs{}
              \PY{p}{[}\PY{n}{n}\PY{o}{*}\PY{l+m+mi}{1}\PY{n}{e4} \PY{k}{for} \PY{n}{n} \PY{o+ow}{in} \PY{n+nb}{range}\PY{p}{(}\PY{l+m+mi}{2}\PY{p}{,}\PY{l+m+mi}{5}\PY{p}{)}\PY{p}{]}
          \PY{n}{puls} \PY{o}{=} \PY{l+m+mi}{2}\PY{o}{*}\PY{n}{np}\PY{o}{.}\PY{n}{pi}\PY{o}{*}\PY{n}{np}\PY{o}{.}\PY{n}{asarray}\PY{p}{(}\PY{n}{freqs}\PY{p}{)} \PY{c+c1}{\PYZsh{} Pulsations associées aux fréquences}
          
          \PY{n}{phases} \PY{o}{=} \PY{p}{[}\PY{l+m+mi}{0}\PY{p}{,}\PY{l+m+mi}{0}\PY{p}{,}\PY{l+m+mf}{0.7}\PY{p}{,}\PY{l+m+mf}{0.8}\PY{p}{,}\PY{l+m+mf}{0.3}\PY{p}{]}
          
          \PY{n}{B\PYZus{}field}\PY{p}{,} \PY{n}{B\PYZus{}function} \PY{o}{=} \PY{n}{create\PYZus{}Bfield}\PY{p}{(}\PY{n}{puls}\PY{p}{,} \PY{n}{phases}\PY{p}{)}
          \PY{n}{B\PYZus{}field}
\end{Verbatim}
\texttt{\color{outcolor}Out[{\color{outcolor}262}]:}
    
    \[\left(J_{0}\left(0.000418879020478639 r\right) - 0.778446653450105 Y_{0}\left(0.000418879020478639 r\right)\right) \cos{\left (125663.706143592 t - 0.7 \right )} + \left(J_{0}\left(0.000628318530717959 r\right) - 0.868179299758213 Y_{0}\left(0.000628318530717959 r\right)\right) \cos{\left (188495.559215388 t - 0.8 \right )} + \left(J_{0}\left(0.000837758040957278 r\right) - 0.92506716196097 Y_{0}\left(0.000837758040957278 r\right)\right) \cos{\left (251327.412287183 t - 0.3 \right )} + \left(J_{0}\left(1.0471975511966 r\right) - 1.0 Y_{0}\left(1.0471975511966 r\right)\right) \cos{\left (314159265.358979 t \right )} + \left(J_{0}\left(1.25663706143592 r\right) - 1.0 Y_{0}\left(1.25663706143592 r\right)\right) \cos{\left (376991118.430775 t \right )}\]

    

    La cellule suivante définit les distances minimale et maximale pour
lesquels tracer le profil du champ magnétique :

    \begin{Verbatim}[commandchars=\\\{\}]
{\color{incolor}In [{\color{incolor}292}]:} \PY{n}{rmin} \PY{o}{=} \PY{l+m+mf}{0.01}
          \PY{n}{rmax} \PY{o}{=} \PY{l+m+mi}{100}
\end{Verbatim}

    \begin{Verbatim}[commandchars=\\\{\}]
{\color{incolor}In [{\color{incolor}313}]:} \PY{n}{times} \PY{o}{=} \PY{p}{[}\PY{l+m+mi}{1}\PY{n}{e}\PY{o}{\PYZhy{}}\PY{l+m+mi}{6}\PY{o}{*}\PY{n}{k} \PY{k}{for} \PY{n}{k} \PY{o+ow}{in} \PY{p}{[}\PY{l+m+mi}{0}\PY{p}{,} \PY{l+m+mi}{10}\PY{p}{,} \PY{l+m+mi}{20}\PY{p}{]}\PY{p}{]}
          
          \PY{n}{fig\PYZus{}Btheta}\PY{p}{,} \PY{n}{ax\PYZus{}Btheta} \PY{o}{=} \PY{n}{graphe\PYZus{}B}\PY{p}{(}\PY{n}{times}\PY{p}{)}
          \PY{n}{fig\PYZus{}Btheta}\PY{o}{.}\PY{n}{savefig}\PY{p}{(}\PY{l+s+s1}{\PYZsq{}}\PY{l+s+s1}{profil\PYZus{}champ.png}\PY{l+s+s1}{\PYZsq{}}\PY{p}{)}
\end{Verbatim}

    \begin{center}
    \adjustimage{max size={0.9\linewidth}{0.9\paperheight}}{variable_current_field_files/variable_current_field_12_0.png}
    \end{center}
    { \hspace*{\fill} \\}
    
    \begin{Verbatim}[commandchars=\\\{\}]
{\color{incolor}In [{\color{incolor}289}]:} \PY{n}{t} \PY{o}{=} \PY{n}{times}\PY{p}{[}\PY{l+m+mi}{2}\PY{p}{]}    \PY{c+c1}{\PYZsh{} Temps auquel calculer le portrait du champ (pas de liste)}
          
          \PY{n}{output\PYZus{}field} \PY{o}{=} \PY{n}{build\PYZus{}field}\PY{p}{(}\PY{n}{t}\PY{p}{)}
          \PY{n}{output\PYZus{}field}\PY{o}{.}\PY{n}{savefig}\PY{p}{(}\PY{l+s+s2}{\PYZdq{}}\PY{l+s+s2}{champmag\PYZus{}courant\PYZus{}variable.png}\PY{l+s+s2}{\PYZdq{}}\PY{p}{)}
\end{Verbatim}

    \begin{center}
    \adjustimage{max size={0.9\linewidth}{0.9\paperheight}}{variable_current_field_files/variable_current_field_13_0.png}
    \end{center}
    { \hspace*{\fill} \\}
    
    \subsection{Animations}\label{animations}

    \emph{work in progress}

L'idée est de réécrire les fonctions de la partie précédente pour
qu'elles produisent un fichier .mp4 qui donne un profil du champ
magnétique évoluant au cours du temps (utiliser la classe
\texttt{animate} de \texttt{matplotlib}).

    \section{Théorie (non-trivial) (BAC +
2)}\label{thuxe9orie-non-trivial-bac-2}

    Le champ magnétique \(\mathbf{B}\) dérive d'un champ \(\mathbf A\)
appelé \emph{potentiel vecteur} :
\(\mathbf{B} = \boldsymbol\nabla\wedge\mathbf{A}\). Par symétrie
cylindrique, on a
\(\mathbf{B}(\mathbf r, t) = B(r,t)\mathbf{e}_\theta\). Par suite
\(\mathbf A(\mathbf r,t) = A(r,t)\mathbf e_z\).

    Le potentiel vecteur \(\mathbf{A} = A(r,t)\mathbf{e}_z\) est solution de
l'équation d'onde

\begin{equation}
\Delta\mathbf A  - \frac{1}{c^2}\frac{\partial^2\mathbf A}{\partial t^2} = - \mu_0\mathbf{J}(r,t),
\end{equation}

avec \(\mathbf J(r,t) = \dfrac{i(t)\delta(r)}{2\pi r}\mathbf{e}_\theta\)
la densité volumique de courant.

Pour un courant sinusoïdal \(i(t) = I\exp(i\omega t)\), le potentiel
s'écrit \(A(r,t) = f(r)\exp(i\omega t)\) et l'équation aux dérivées
partielles se réduit à

\begin{equation}
\frac{1}{r}\frac{\mathrm d}{\mathrm dr}\left(r\frac{\mathrm df}{\mathrm dr} \right) + k^2f(r) = -\frac{\mu_0I\delta(r)}{2\pi r},
\end{equation}

avec \(k=\dfrac{\omega}{c}\).

La solution générale prend la forme

\[
f(r) = CJ_0(kr) + DY_0(kr)
\]

où \(C\) et \(D\) dépendent de la pulsation \(\omega\) du courant, et
\(J_0,Y_0\) sont les 0-ièmes fonctions de Bessel de la première et
seconde espèce, solutions de \[xy''(x) + y'(x) + xy(x) = 0.\]


    % Add a bibliography block to the postdoc
    
    
    
    \end{document}
