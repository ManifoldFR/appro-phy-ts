\documentclass{article}
\usepackage{polyglossia}
\usepackage[a4paper]{geometry}
\usepackage{mathtools, amssymb, amsfonts}
\usepackage{amsthm}
\usepackage{fontspec}
\usepackage{titling}
\usepackage{listings}
\usepackage{graphicx}
\usepackage[hidelinks]{hyperref}
\usepackage{tikz}
\usepackage[shortlabels]{enumitem}
\usepackage{comment}

\setdefaultlanguage{french}

%% Math commands
\renewcommand\epsilon\varepsilon
\renewcommand\phi\varphi
\newcommand{\N}{\mathbb N}
\newcommand{\Z}{\mathbb Z}
\newcommand{\R}{\mathbb R}
\newcommand{\Q}{\mathbb Q}
\newcommand{\CC}{\mathbb C}
\newcommand{\K}{\mathbb K}
\DeclarePairedDelimiter{\abs}{\lvert}{\rvert}

%% Titling configuration
\pretitle{\begin{center}\LARGE}
\title{\textsc{Effet Doppler}}
\posttitle{\par\end{center}\vspace{-1.2em}}

\preauthor{}
\author{}
\postauthor{}

\date{\today}

\begin{document}
\maketitle

\begin{abstract}
	L'effet Doppler est le phénomène par lequel la perception d'un signal (acoustique, électromagnétique) émis par une source en mouvement relativement à un observateur présente, pour un signal monochromatique composé d'une seule fréquence, un décalage de celle-ci -- ou, pour un signal composé, un décalage de son spectre en fréquence.
	
	On va ici aller plus loin dans l'explication de ce phénomène, et démontrer la formule donnée dans le cours.
\end{abstract}

Vous connaissez certainement le principe de la décomposition de Fourier: on peut écrire tout signal comme somme de signaux monochromatiques, de la forme $A(\omega)\cos(\omega t - \phi(\omega))$, où $A(\omega)$ est l'amplitude de l'harmonique $\omega$ du signal initial et $\phi(\omega)$ sa phase. Vous avez mis en application ce principe en TP, lorsque vous avez réalisé à l'aide d'un logiciel les spectres en fréquence -- concrètement, les courbes de la fonction $A(\omega)$ -- de plusieurs signaux.

Pour cette raison, on peut simplifier le problème et se placer dans le cas où le signal émis est monochromatique, c'est-à-dire composé d'une seule longueur d'onde $\lambda = c/f$, où $c$ est la célérité des ondes dans le milieu et $f$ la fréquence correspondante.


Il se comprend de la façon suivante : entre chaque émission d'un front d'onde s'écoule une durée égale à la période $T=\lambda/c=1/f$ de la source, celle-ci se déplace de $\vec v T$.

\begin{center}
\begin{tikzpicture}
\draw [thick, blue, ->] (0,0)--(2,1);
\node [below left] (S) at (0,0) {Source};
\coordinate at (0,0);
\end{tikzpicture}
\end{center}

Si $\vec u$ représente la ligne de visée de l'observateur, la distance entre les fronts d'ondes émis entre $t$ et $t+T$ est $\tilde{\lambda} = (c+\vec v\cdot\vec u)T$, de sorte que la fréquence du signal perçu est

\[
\tilde{f} = \frac{c}{\tilde{\lambda}} = \frac{c}{(c+\vec u \cdot \vec u)T} = \frac{c}{c+\vec v\cdot\vec u}f
\]

\end{document}